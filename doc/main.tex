\documentclass[conference]{IEEEtran}
\IEEEoverridecommandlockouts
% The preceding line is only needed to identify funding in the first footnote. If that is unneeded, please comment it out.
\usepackage{ctex}
\usepackage{cite}
\usepackage{amsmath,amssymb,amsfonts}
\usepackage{algorithmic}
\usepackage{graphicx}
\usepackage{textcomp}
\usepackage{xcolor}
\def\BibTeX{{\rm B\kern-.05em{\sc i\kern-.025em b}\kern-.08em
    T\kern-.1667em\lower.7ex\hbox{E}\kern-.125emX}}
\begin{document}


% \title{文章标题}

% \author{
% \IEEEauthorblockN{张三}
% \and
% \IEEEauthorblockN{李四}
% }

% \maketitle

\input{title.tex}

\begin{abstract}
论文摘要是对文章内容不加诠释和评论的简单陈述。一般控制在 200 字左右,建议在论文全部完成后再动手写摘要。
\end{abstract}


\section{引言(飞)}
引言小节主要负责写清楚以下问题:
\begin{enumerate}
\item 本文要解决什么问题
\item 这个问题为什么需要解决
\item 其他人是如何解决这个问题的,有何不足
\item 简单说一下本文解决问题的思路
\end{enumerate}

\section{相关工作(飞)}
相关工作小节主要写和本文主题密切相关的技术/知识点,方便第三节的阐述。

比如本文需要做一个图像识别算法,那本节可以介绍常见的图像识别算法;或者本文主要基于方法 A 进行改进,本节可以对方法 A 进行介绍。

\subsection{子小节}

一般从第二节开始,会出现子小节,请按照逻辑合理组织论文。

\section{论文的主要工作}
本节的标题需要自拟。


\subsection{Socket通信与数据包}
\subsubsection{连接}
\subsubsection{收发包}

\subsection{场景物体}
\subsubsection{资源管理}
\subsubsection{创建}
\subsubsection{销毁}
\subsubsection{同步}
\subsubsection{物理}


\subsection{玩家角色}
\subsubsection{移动}
\subsubsection{蓄力}
\subsubsection{大招}
\subsubsection{动画}
\subsubsection{队伍、球门和得分}


\subsection{游戏效果}
\subsubsection{玩家名字}
\subsubsection{小地图}
\subsubsection{后期处理}

\subsection{实用代码}
\subsubsection{单例爷爷}
\subsubsection{线程管理}


% \subsection{游戏逻辑}
% \subsubsection{物体基类}
% \subsubsection{资源管理}
% \subsubsection{物体创建}
% \subsubsection{玩家角色}
% \subsubsection{特性}

% \subsection{服务器}
% \subsubsection{物理}
% \subsubsection{控制玩家角色}
% \subsubsection{大招}
% \subsubsection{状态同步}

% \subsection{客户端}
% \subsubsection{相机追踪}
% \subsubsection{玩家输入}
% \subsubsection{后期特效}
% \subsubsection{蓄力}





本节需要详细介绍本文提出/设计的方法。

公式示例:
\begin{equation}
a+b=\gamma\label{eq:demo}
\end{equation}
式~(\ref{eq:demo})是一个演示用的公式。

表格示例。
\begin{table}[htbp]
\caption{表格}
\begin{center}
\begin{tabular}{|c|c|c|c|}
\hline
\textbf{Table}&\multicolumn{3}{|c|}{\textbf{Table Column Head}} \\
\cline{2-4} 
\textbf{Head} & \textbf{\textit{Table column subhead}}& \textbf{\textit{Subhead}}& \textbf{\textit{Subhead}} \\
\hline
copy& More table copy$^{\mathrm{a}}$& &  \\
\hline
\multicolumn{4}{l}{$^{\mathrm{a}}$Sample of a Table footnote.}
\end{tabular}
\label{tbl:test}
\end{center}
\end{table}
表格的代码可以实现在 http://www.tablesgenerator.com/ 上设计好,然后复制过来。表格~\ref{tbl:test}是一个示例。

图片的示例:
\begin{figure}[htbp]
\centerline{\includegraphics{images/fig1.png}}
\caption{Example of a figure caption.}
\label{fig:figure1}
\end{figure}
图~\ref{fig:figure1}是一个点。在表格和图中使用 label 可以指定标签,便于后续使用 ref 命令引用。

参考文献\cite{IEEEexample:softonline,IEEEexample:tamethebeast,IEEEexample:standard}需要先将参考文献的信息写入一个 bib 文件,然后在最后一小节中使用 bibliography 命令指定 bib 文件\cite{IEEEexample:urlsty},具体参见tex源文件。

正文中通过使用 cite 命令引用参考文献。


\section{实验结果(飞)}
本节题目可以自拟。

主要负责成果展示。


\section{结论}
对整个工作做一个总结,得出结论,并展望未来。
提升游戏性,性能优化(搞一个打开后就是6960的,没有图形界面的,没准服务器就能跑动了)


\section*{References}

\bibliography{IEEEexample}  
\bibliographystyle{IEEEtran}

\end{document}
